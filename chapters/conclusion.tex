\chapter*{Заключение} \addcontentsline{toc}{chapter}{Заключение} В рамках данной
работы было выполнено прикладное исследование системы интернет поиска nutch. В
качестве изначальной цели работы была поставлена оптимизация работы nutch, с
учетом дополнительной информации о содержании ссылок.

В ходе работы была изучена архитектура nutch и проанализированы особенности его работы,
что позволило выявить следующие направления оптимизации процесса сбора страниц с ресурсов 
сети Интернет:
\begin{itemize} 
\item изменение алгоритма ранжирования ссылок в очереди на скачивание, учитывающего .... .
\item автоматическое создание черных списков фильтров url, которые не
    должны скачиваться. 
\end{itemize}

Были написаны необходимые расширения nutch и дополнительное приложение для
генерации фильтров на языке Java. В ходе разработки были использованы следующие
технологии: Apache Hadoop, Spring Framework, iBatis, Nutch plugin.

Проведенные эксперименты, а также опыт эксплуатации, показали эффективность реализованого 
алгоритма ранжирования, при котором отношение "полезных" документов к скаченным 
значительно возрастает.

Было показано, что разработанные алгоритмы в совокупности позволяют достичь
высокой эффективности как на ранних стадиях сборки (ранжирование), так и на
поздних (генерация фильтров).
