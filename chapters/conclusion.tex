\chapter*{Заключение}
\addcontentsline{toc}{chapter}{Заключение}
В рамках данной работы было выполнено прикладное исследование системы интернет поиска nutch. В качестве изначальной цели работы была поставлена оптимизация работы nutch, с учетом дополнительной информации о содержании ссылок.

В ходе работы была изучена архитектура nutch. На основе этой информации были найдены основные направления для оптимизации работы:
\begin{itemize}
 \item изменение ранжирования
 \item автоматическое создание фильтров url
\end{itemize}

Были написаны необходимые расширения nutch и дополнительное приложение для генерации фильтров на языке Java. В ходе разработки были использованы следующие технологии: hadoop, spring, ibatis, nutch plugin. 

Были произведены эксперименты и получены количественные данные, характеризующие эффективность нового ранжирования. Было показано что даже простое изменение ранжирования с учетом дополнительной информации может на порядок увеличить производительность.

Было показано, что разработанные алгоритмы в совокупности позволяют достичь высокой эффективности как на ранних стадиях сборки (ранжирование), так и на поздних (генерация фильтров).
