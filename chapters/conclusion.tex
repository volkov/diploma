\chapter*{Заключение} \fixme{перенести в конец. Оставить, просто изменить время (см п 1)}
\addcontentsline{toc}{chapter}{Заключение} 
\section*{Результаты}
\begin{itemize}
 \item Сделан сравнительный анализ различных open source поисковые роботы (DataparkSearch,
AppSeek, mnlGoSearch, Nutch, Hounder, Heritix) и выбрать наиболее подходящий для
решения задачи
 \item Изменено поведение ядра nutch для более эффективной работы с индексом большого объема.
 \item Проанализированы различные Key Value хранилища (Memcached, MongoDb, Project Voldemort, Tokyo Cabinet) и выбрано MongoDb в качестве хранилища для системы удаления дубликатов из индекса
 \item Разработан и реализован плагин к Nutch для раннего удаления дубликатов
 \item Разработан и реализован плагин для более эффективного ранжирования ссылок для новостных сайтов
 \item Разработана и реализована система для автоматического создания URL фильтров
 \item Измененный поисковый робот протестирован на реальных данных.
\end{itemize}
В ходе данной работы был создан поисковый робот, способный скачивать и индексировать до 80 000 новостных документов в день, использующий кластер из трех машин ($2\times2,5$ GHz 2007 Xeon, 1,7GB RAM). Было показано, что специализация поискового робота на определенной части Web может существенно увеличить его произодительность.