\chapter*{Введение}
\addcontentsline{toc}{chapter}{Введение} 
\epigraph{— В то время, когда наши корабли бороздят просторы
Вселенной…}{} На текущем этапе развития, когда общество осуществляет переход от
постиндустриальной эпохи к информационной, требования к системам хранения и
обработки информации непрерывно растут. Традиционные подходы не справляются с
ростом количетва данных. Трудно оценить общий объем данных, однако, по оценкам
IDC (International Data Corporation) в данный момент хранится порядка
$1.8\cdot10^{21}$ байт, что в 10 раз больше чем в 2006 году.

К значительному количеству данных можно получить доступ через Всемирную Паутину
(www). При таких объемах остро стоит задача организации эффективного поиска. Уже
в 2009 году Google Search обработал более 109,5 миллионов сайтов, и более
$10^{12}$ уникальных URL. На данный момент их индекс содержит $4\cdot10^{10}$ документов.

Одной из специфических областей поиска является поиск по новостным ресурсам.
 Для документов с новостных сайтов характерна привязка к дате, региону и тематике.
 Таким образом такие документы легко классифицировать, что позволяет производить более качественный поиск и анализ.
 В качественном инструменте для анализа СМИ заинтереснованы различные консалтинговые и pr агенства, пресс-службы, маркетинговые отделы крупных компаний.

Одна из задач поисковой системы - нахождение и загрузка документов(Web crawling), за которую отвечает поисковый робот(Spider, Crawler).
 Web crawling весьма ресурсоемкий процесс. Основные проблемы связаны с большим количеством данных,
 остутсвием  контроля над данными, постоянным изменением структуры ресурсов, динамическим созданием страниц и низким качеством некоторых ресурсов.
 Однако, специализация на определенной узкой области web позволяет существенно повысить производительность web crawler'а.

Конечной целью работы является создание системы способной эффективно индексировать новости в рунете.

\chapter{Обзор области}
\section{Web Search} 
\textbf{Поисковая система} --- система, разработанная для
поиска информации в www. Результаты поиска которой, как правило, представлены в
виде списка ``попадений''\fixme{Не понятно что такое попадения}. Информация может состоять из веб страниц,  изображений,
мультимедийной информации. Одной из первых поисковых систем стал
проект Archie, разработанный в 1990 году студентами McGill University. Программа скачивала
списки файлов с открытых ftp серверов, и добавляла их в базу с возможностью
поиска по названию. %Дальнейшим 

Поисковая система состоит из трех основных компонент: 
\begin{itemize}
 \item поисковый робот --- программа, предназначенная для перебора докуметнов и
занесения данных о них в базу. 
 \item индексатор --- программа, создающая на основе полученных с помощью робота данных индекс.
 \item поисковик --- программа, осуществляющая поиск в полученном индексе на основе поискового запроса.
\end{itemize} В условиях постоянно расширяющегося и изменяющегося www, непрерывно
возрастают требования к поисковым системам. 

\paragraph{Системы общего поиска} нацелены на охват большей части данных
доступных в www. Такие системы предназначены для поиска наиболее релевантных
документов относящихся к объекту поиска. 
\paragraph{Системы тематического поиска} более разнообразны, и требования к ним более специфичны.
 Например Google Microblogging Search Engine, ориентированный на поиск по записям в микроблогах,
где крайне важна задержка между созданием записи, и ее попадением в индекс.

\section{Поиск по новостям}

Основные источники новостей в www --- это электронные СМИ и блоги. По данным
liveinternet на 2008 год, рунет насчитывает 4392 сайта СМИ,
 а число блогов значительно больше - по данным Яндекс за 2009 год в русскоязычной блогосфере насчитывается порядка 840000 активных блогов, 
на которых ежедневно публикуется порядка 300000 постов.\footnote{http://mediarevolution.ru/audience/1962.html}
\fixme{Дополнить данными по нашему проекту. Посмотреть, сколько мы считываем URL'ов по RSS}
Очевидно, за прошедшее время количество таких сайтов значительно увеличилось. За сутки каждое
из подобный изданий публикует до 100 документов(lenta.ru). Таким образом, 
можно говорить о десятках миллионов создаваемых документов в год.

Под новостью понимается документ содержащий текст, заголовок и дату. Для СМИ и
блогов характерно:
\begin{itemize} 
 \item большое количество посторонних страниц, не содержащих новостей;
 \item схожая структура (как именования url, так и самого html);
 \item наличие rss ленты.
\end{itemize}

К новостным поисковым системам предъявляются следущие требования:
\begin{itemize} 
\item минимальное время между публикацией статьи на новостном ресурсе и ее 
    предоставление в поисковой выдаче;
\item поик должен осуществлять не по всей HTML--странице, а только по ее 
    существенным частям. 
\end{itemize}

\section{Постановка Задачи}
Конечной целью работы является создание поискового робота способного эффективно индексировать новости в рунете.

\textit{Поисковый робот}(Web crawler) --- программа для поиска веб-страниц в сети\cite{crawl}. Грубо говоря поисковый робот начинает с URL для начальной страницы $p_{0}$.
 Он скачивает $p_{0}$, выделяет все URL которые в ней находятся, и добавляет их в очередь URL (\textit{crawling frontier}). Затем робот в некотором порядке выбирает URL из очереди и повторяет процесс.

Каждая скачанная страница передается клиенту, котрый затем создает индекс по страницам.
\subsection{Условия}
Важным фактором, влияющим на качество поиска, является идентификация страницы содержащей новость
 и выделение её содержательной части. В данной работе предполагается наличие базы данных,
 содержащей правила на основе регулярных выражений, которые по URL определяют содержит ли данная страница новость,
 а так же правил по которым из веб-страницы выделяется содержательная часть.

Далее под \textit{документом} понимаются выделенные по этим правилам данные.

\subsection{Требования}
\begin{itemize}
 \item поддержка десятков миллионов документов;
 \item скорость роста базы документов --- более 50 тысяч в день;
 \item попадение в индекс документов из rss лент не позже чем через 12 часов после их публикации;
 \item попадение в индекс старых документов из архива --- некоторые документы могут находиться достаточно ``глубоко`` (например что-бы получить новости месячной давности на ресурсе fontanka.ru необходимо сделать 5 переходов).
 \item отсутствие дубликатов в пределах одного домена --- под дубликатом понимается документ у которого совпадает URL или текст с другим документом.
\end{itemize}

В качестве основного показателя эффективности используется количество полученных документов за сутки.
 Поскольку объем данных непрерывно увеличиватся, неизбежна деградация производительности.
 Поэтому также в качестве метрики используется отношение количества полученных за сутки документов к общему числу документов.



\chapter{Обзор средств}
Большинство популярных поисковых сервисов предоставляют возможность поиска по новостям (google, yandex, yahoo!), однако они пользуются закрытыми алгоритмами и не предоставляют доступа непосредственно к индексу
\section{Сравнение open source поисковых роботов}
Существует достаточно много open source поисковых роботов. Для успешного решения задачи робот должен справлятся с нагрузкой (порядка 100000 документов в день, база ссылок порядка $10^{9}$ и порядка $10^7$ документов в индексе), быть легко изменяем и расширяем. Поскольку предполагается коммерческое использование робота не протяжении долгого времени, проект должен быть достаточно зрелым и развивающимся.
\paragraph{Метрики}
\begin{itemize}
 \item язык
 \item поддержка robots.txt
 \item распределенность системы
 \item тип хранения индекса
 \item тип хранилища url
 \item поддержка
\end{itemize}
\paragraph{Роботы}
\begin{itemize}
 \item DataparkSearch --- поисковая система разработаная для поиска по локальным файлам, группам сайтов и интранету
 \item AspSeek --- поисковая система оптимизированная для работы с многими сайтами, и средней загрузкой --- до нескольких миллионов страниц
%  \item mnoGoSearch --- yet another crawler in C
 \item Nutch --- поисковая система основанная на Lucene
 \item Hounder --- поисковая система онованная на nutch
 \item Heritix --- еще одна java поисковая система
\end{itemize}
\paragraph{Сравнение}
\begin{table}[h]
\caption{\label{tab:crawlers}Сравнение поисковых роботов.}
\begin{center}
\begin{turn}{90}
\begin{tabular}{|c|c|c|c|c|c|c|}
\hline
Название & Язык & Распределенность & robots.txt & Индекс & Хранилище url & Документы\\
\hline
DataparkSearch & C & ~ & + & SQL database/собственный формат & SQL database & $10^{6}$\\
\hline
AspSeek & C++ & ?? & + & SQL database & SQL database & $10^{6}$\\
\hline
Nutch & Java & + & + & Lucene index & распределенный файл & $10^{9}$\\
\hline
Hounder & Java & + & + & Lucene index & распределенный файл & ???\\
\hline
\end{tabular}
\end{turn}
\end{center}
\end{table}


\subsection{Описание роботов}
\paragraph{DataparkSearch}\footnote{\href{http://www.dataparksearch.org/}{http://www.dataparksearch.org/}} --- предназначен для работы с небольшой группой сайтов или интранета, написан на C. Состоит из двух частей --- индексатора и CGI фронтенда. DataparkSearch отделился в 2003 году от mnoGoSearch. Имеет встроенные парсеры для html, xml, есть возможность написания собственных парсеров для других форматов. Данные по ссылкам хрянятся в SQL базе данных. Можно запустить сразу несколько процессов индексации работающих с одной базой. Данные по документам могут храниться как в бд, так и в собственном формате на диске (cache mode), который эффективно работает с несколькими миллионами документов.
\paragraph{AspSeek}\footnote{\href{http://www.aspseek.org/}{http://www.aspseek.org/}} --- поисковая система написанная на C++ и оптимизированная для работы с множеством сайтов. Состоит из индексирующего робота, поискового демона и CGI фронтенда. Данные поискового сервера хранятся в SQL базе данных и бинарных файлах (delta files), рассчитан для работы с несколькими миллионами документов.
\paragraph{Nutch}\footnote{\href{http://nutch.apache.org/}{http://nutch.apache.org/}} --- поисковый робот написанный на java, работающий поверх системы Hadoop\footnote{\href{http://hadoop.apache.org/}{http://hadoop.apache.org/}}. Изначально Nutch разрабатывался в рамках проекта Lucene\footnote{\href{http://lucene.apache.org/}{http://lucene.apache.org/}}, однако в 2005 году отделился как отдельный проект. Благодаря работе поверх Hadoop обладает хорошей маштабируемостью (до 100 машин в кластере). Nutch отличается гибкой системой плагинов, через которые осуществляется поддрежка множества протоколов (http, ftp, file) и форматов (от html до msexcel и swf).
\paragraph{Hounder}\footnote{\href{http://hounder.org/}{http://hounder.org/}} --- поисковая система на java, робот которой основан на Nutch. Из дополнительного функционала следует отметить фильтр Байеса для разбиения документов по категориям.

\subsection{Выбор}
В качестве основы системы был выбран Nutch, так как он полностью удовлетворяет требованиям:
\begin{itemize}
 \item нагрузка --- Nutch использовался в качестве основы для Sapphire Web Crawler\footnote{\href{http://boston.lti.cs.cmu.edu/crawler/index.html}{http://boston.lti.cs.cmu.edu/crawler/index.html}}, с помощью которого было скачано более $10^{9}$ документов со средней скростью в 431 документ в секунду.
 \item расширямеость --- благодаря модульности и гибкой системе плагинов можно достаточно легко изменять поведение системы.
 \item поддержка --- проект разрабатывается более 7 лет, текущая стабильная версия проекта 1.2 была выпущена в сентябре 2010. Проект поддерживается ``Yahoo! Research Labs''.
\end{itemize}

\section{Архитектура Nutch}
Высокая масштабируемость робота достигается за счет работы поверх MapReduce фреймворка \textit{Hadoop}\cite{hadoopdefguide}. \textit{Hadoop} на данный момент представляет набор подпроектов Apache Software Foundation, среди которых находятся Hadoop MapReduce и HDFS. 
\paragraph{MapReduce} --- модель программирования для обработки больших объемов данных, впревые опубликованная\cite{googlemr} Google в 2004 году. При данном подходе логика программы реализуется в функциях \textit{map}, которая преобразует пары ключ/значение в набор промежуточных пар ключ/значение, и \textit{reduce}, которая обрабатывает все значения связанные с одним промежуточным ключом \ref{eq:mapred}.
\begin{equation}\label{eq:mapred}
\begin{split}
map:\langle key_{in}, value_{in}\rangle\rightarrow\langle key_{int}, value_{int}\rangle^{*} \\
reduce:\langle key_{int}, value_{int}^{+}\rangle\rightarrow\langle key_{out}, value_{out}\rangle^{*}
\end{split}
\end{equation}
Написанная таким образом программа может автоматически параллельно выполняться на кластере машин, программное обеспечение которых брало бы на себя распределение данных, управление выполнением задач, поддержку отказов и управление взаимодействием между узлами кластера. 
\paragraph{HDFS (Hadoop Distributed File System)} --- распределенная система хранения данных, основанная на модели GFS\cite{gfs} --- распределенной файловой системы используемой Google. \textit{HDFS} предназначена для:
\begin{itemize}
 \item больших файлов --- имеются ввиду файлы от нескольких сотен мегабайт до нескольких террабайт;
 \item потокового доступа к данным --- предпологается что данные записываются один раз, и программа в процессе работы использует большую часть из записанного набора данных;
 \item дешевого железа --- риск отказа оборудования достаточно высок.
\end{itemize}
Файл в \textit{HDFS} представляет из себя последовательность достаточно больших блоков (по умолчанию 64Mb), которые в нескольких экземплярах (обычно используется 3 реплики) хранятся на различных узлах ---\textit{DataNode}. Последовательность блоков в файле и их расположение управляется через \textit{NameNode}. Таким образом нагрузка на передачу и запись данных распределяется между набором \textit{DataNode}, а структура папок и файлов находится в \textit{NameNode}.

\subsection{Основные Этапы}
\textit{Nutch} является средством инкрементальной сборки, на каждом этапе выполняются следующие действия:
\begin{itemize}
 \item \textit{inject} --- добавление списка URL в базу ссылок \textit{crawldb} (используется при инициализации сборки или при добавлении новых доменов);
 \item \textit{generate} --- из \textit{crawldb} выбирается фиксированное число ссылок для их последующиего скачивания;
 \item \textit{fetch} --- скачивание документов по выбранным ссылкам;
 \item \textit{parse} --- документы парсятся, выделяются ссылки;
 \item \textit{invertlinks} --- обновляется база обратных ссылок;
 \item \textit{index} --- создается индекс по сегменту;
 \item \textit{merge index} --- индекс по сегменту объединяется с основным;
 \item \textit{update} --- обновляется \textit{crawldb}
\end{itemize}

\subsection{Система Плагинов}
Особого внимания заслуживает система плагинов в \textit{Nutch}, именно через плагины реализована основная функциональность. Плагины выполняют разбор документов, индексацию, поиск, ранжирование, фильтрацию ссылок и.т.д.

Каждый плагин предоставляет одно или несколько \textit{расширений} (\textit{extensions}) для \textit{точек расширений} (\textit{extension points}), причем сами по себе \textit{точки расширений} определены в плагине.
