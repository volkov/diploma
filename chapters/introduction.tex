\chapter{Введение}
\epigraph{— В то время, когда наши корабли бороздят просторы Вселенной…}{}
На текущем этапе развития, когда общество осуществляет переход от постиндустриальной эпохи к информационной, требования к системам хранения и обработки информации непрерывно растут. Традиционные подходы не справляются с ростом количетва данных. Трудно оценить общий объем данных, однако, по оценкам IDC (International Data Corporation) в данный момент хранится порядка $1.8\cdot10^{21}$ байт, что в 10 раз больше чем в 2006 году.

И тут надо как то ввернуть про Всемирную паутину, которая безусловно имеет некое отношение ко всему выше сказанному. Уже в 2009 году Google Search обработал более 109,5 миллионов сайтов, и более $10^{12}$ уникальных url.

Бла бла бла новости, бла бла бла очень много...

\section{Web Search}
\textbf{Поисковая система} --- система разработанная для поиска информации в www. Результаты поиска которой, как правило, представлены в виде списка ``попадений''. Информация может состоять из веб страниц или их частей, изображений, видо e.t.c. Пожалуй, самой превой поисковой системой был Archie, созданный в 1990 году студентами McGill University. Программа скачивала списки файлов с открытых ftp серверов, и добавляла их в базу с возможностью поиска по названию.

Структуру поисковой системы можно разбить на три части:
\begin{itemize}
 \item поисковый робот --- программа предназначенная для перебора докуметнов, и занесения данных о них в базу.
 \item индексатор --- программа создающая на основе полученных с помощью робота данных индекс.
 \item поисковик --- программа осуществляющая поиск в полученном индексе на основе поискового запроса.
\end{itemize}
 
В условиях постоянно расширяющегося и изменяющегося www, возрастают требования к поисковым системам. Каждый год стартует порядка 3-5 новых поисковых систем. Поисковые системы можно разделить на два типа: системы общего поиска, и системы тематического поиска.

\paragraph{Системы общего поиска} нацелены на охват большей части данных доступных в www. Такие системы предназначены для поиска наиболее релевантных документов относящихся к объекту поиска.
\paragraph{Системы тематического поиска} более разнообразны, и требования к ним более специфичны. Например Google Microblogging Search Engine, ориентированный на поиск по записям в микроблогах, где крайне важна задержка между созданием записи, и ее попадением в индекс. Далее будут более подробно рассмотрены особенности поиска по новостям.

\section{Поиск по новостям}

Основные источники новостей в www это интернет издания и блоги. По данным liveinternet на 2008 год, рунет насчитывает 4392 сайта СМИ. Очевидно, за прошедшее время количество таких сайтов значительно увеличилось. За сутки каждое из подобный изданий публикует до 100 документов(lenta.ru). Если предположить, что в среднем публикуется по 5 новостей на издание в день, то за год создается более 8 миллионов русскоязычных статей. Разумно рассчитывать на десятки, если не сотни миллионов документов в этой области.

Под новостью понимается документ содержащий текст, заголовок и дату. Для СМИ и блогов характерно:
\begin{itemize}
 \item большое количество посторонних страниц, не содержащих новостей
 \item схожая структура (как именования url, так и самого html)
 \item наличие rss ленты
\end{itemize}


Для новостных поисковых систем важны следующие требования:
\begin{itemize}
 \item время от публикации до попадения в индекс должно быть как можно меньше (не на столько как в поиске по микроблогам, но значительно меньше чем при сборке обычных документов)
 \item поиск должен осуществляться именно по новости, а не по странице ее содержащей
\end{itemize}
